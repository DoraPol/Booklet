
    \begin{abstract_online}{Synchronization of coupled active rotators by common noise}{%
        \underline{A.V. Dolmatova}$^{1}$, D.S. Goldobin$^{1,2}$, A. Pikovsky$^{3}$}{%
        }{%
        $^1$ Institute of Continuous Media Mechanics, UB RAS, Perm, Russia\newline{}$^2$ Department of Theoretical Physics, Perm State University, Russia\newline{}$^3$ Institute for Physics and Astronomy, University of Potsdam, Germany}
    We study the effect of common noise on coupled active rotators. While such a noise always  facilitates synchrony, coupling may be attractive (synchronizing) or repulsive (desynchronizing).  On the basis of the Ott-Antonsen theory, one can derive the finite-dimensional equation system  governing the collective dynamics of the system for the case of global coupling. We develop an  analytical approach based on a transformation to approximate angle-action variables and  averaging over fast rotations. For identical rotators, we describe a transition from full to partial synchrony at a critical value of repulsive coupling. For nonidentical rotators, the most nontrivial  effect occurs at moderate repulsive coupling, where a juxtaposition of phase locking with  frequency repulsion (anti-entrainment) is observed. We show that the frequency repulsion  obeys a nontrivial power law.  A.V.D. and D.S.G. acknowledge financial support by the Russian Science Foundation (Grant No.  -12-00090)  
    
    \end{abstract_online}
    