
    \begin{abstract_online}{A Special Talk about Diversity in Science}{%
        F. Schmidt}{%
        \ITtag}{%
        Physics Department, Technical University of Munich, Germany}
    One of the most consistent global trends in science over the last decades concerns the move from individual efforts of scientists to collaborative projects among several researchers. Across most scientific disciplines the solitary researcher has been replaced by collective efforts between individuals often distributed across organizations and countries. Accompanying this trend are efforts to understand the “ingredients” of excellent research teams. Social sciences, but also social psychology as well as social networks analysis have all contributed to the understanding of successful collaborations. This contribution will review the elements of excellent teams as well as highlighting how bias and stereotypes can undermine optimal group performance. The presentation will be rounded up by examples how diversity not only affects research collaboration but also improves the quality of the produced science and engineering knowledge. 
    
    \end{abstract_online}
    